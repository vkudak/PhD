\chapter{Kepler data processing}
\label{Chapter3}

\section{Fitting eclipsing binary minima}
To obtain precise time of minima in eclipsing binary systems we need to define function that will fit minima shape with best precision.
There is phenomenological and physical methods that allow fo fit all light curve. Because we have a large number of eclipsing binary systems in Kepler EB catalogue we prefer to use phenomenological fitting of only minima parts of light curves.

\subsection{Gauss and Lorentzian fit}
The main advantage of this two function is their simplicity and ability to get coordinates of centre ($x_{c}$) after fitting light curve.
With another more complex functions that are discussed below we must to derivative it before we can find $x_{c}$.

Gaussian function:
\begin{equation}\label{eq:gaus}
F(\varphi)= A\cdot \exp(-\frac{(\varphi - x_{c})^2}{2\sigma^2})
\end{equation}

Lorentzian function: 
\begin{equation}\label{eq:lorentz}
F(\varphi)= A \frac{w^2}{w^2 + (\varphi - x_{c})^2}
\end{equation}

Functions perform good fit when EB minima has shape similar to normal distribution. In case of narrow minima such functions have a slightly larger error. Also function perform good fit when we fit only minima part but not all light curve. 

\subsection{Mikulasek phenomenological model}
Method is aimed to establish a general model of LCs of eclipsing
systems (ES; both eclipsing binaries and stars with transiting
planets) that could fit the LCs with an accuracy of 1\% of their
amplitudes or better. 

The model function of a monochromatic light curve (expressed in magnitudes) of eclipsing
systems $F(\varphi, \lambda)$, can be assumed as the sum of three particular functions:

\begin{equation} \label{eq:mik_general}
F(\varphi, \lambda) = F_{e}(\varphi, \lambda) + F_{p}(\varphi, \lambda) + F_{c}(\varphi, \lambda)
\end{equation}

where $F_{e}$ describes the mutual eclipses of the components, $F_{p}$ models the proximity effects,
while $F_{c}$ approximates the O’Connell effect (irrespectively of its physical cause)

The profiles of both minima
are complex functions determined primarily by the geometry
of the system and the relative brightness of components in
a given spectral region centred at the effective wavelength $\lambda_{eff}$.
The contribution of eclipses $F_{e}(\vartheta, \lambda_{eff})$ to an ES light curve can
be approximated by a sum of two special periodic functions of
phase function $\vartheta$. In the case of circular orbits, eclipses are exactly
symmetrical around their centres at phases $\varphi_{01}$ and $\varphi_{02}$. If
we put the origin of the phase function $M_{0}$ at the time of the
primary minimum, then $\varphi_{01} = 0$, $\varphi_{02} = 0.5$.
The model function was selected so that it describes as aptly
as possible those parts of LCs that are in the vicinity of their
inflex points, where their slopes are maximum. The functions
are parameterized by their widths $D_{1}, D_{2}$, eclipse LC kurtosis
coefficients $\Gamma_{1}, \Gamma_{2}$, dimensionless correcting factors $C_{1},C_{2}$, and
central depths $A_{1}(\lambda_{eff}), A_{2}(\lambda_{eff})$:

\begin{equation} \label{eq:mik_main}
F_{e}(\vartheta, \lambda_{eff})=\sum_{k=1}^{n_{e}} A_{k} \left( 1+C_{k} \frac{\varphi_{k}^2}{D_{k}^2}\right) 
\left\lbrace 1-\left\lbrace 
1-exp\left[ 1-\cosh\left(\frac{\varphi_{k}}{D_{k}}\right)\right] 
\right\rbrace^{\Gamma_{k}}\right\rbrace 
\end{equation}

\begin{equation} \label{eq:mik_2}
\varphi_{k} = \vartheta - 0.5 (k - 1) - round \left[ \vartheta - 0.5 (k - 1)\right] ,
\end{equation}

where the summation is over the number of eclipses during one
cycle, $n_{e}$: $n_{e} = 2$ or $n_{e} = 1$ (the common situation for exoplanet
transits). Each eclipse in a given colour is thus described by only
four parameters – its depth $A$, width $D$, kurtosis $\Gamma$, and the correcting
parameter $C$. \parencite{mikulasek2015}

In the case of EBs with two minima in a cycle ($n_{e} = 2$), we
need eight parameters, but sometimes the number of needed parameters
can be smaller. Inspecting the parameters $D, \Gamma$, and
$C$ for both eclipses of many EBs, we have concluded that they
are as a rule nearly the same: especially $D1 \cong D2, \Gamma_{1} \cong \Gamma_{2}$, and
$C_{1} \cong C_{2}$. We therefore usually need only five monochromatic parameters
($A_{1}, A_{2}, D, \Gamma,C$). The parameter $C$ is mostly comparable
to its uncertainty, so we can neglect it entirely. Then we need
just four parameters! On the other hand, in EBs with totalities
we see that the bottoms of their occultations are flat whilst transits
are convex. It can be described by introducing of different
parameters $C_{1},C_{2}$.

The LCs of the exoplanet transits ($n_{e} = 1$) need only
four parameters ($A, D, \Gamma,C$; in cases of very precise
measurements, we add another dimensionless parameter $K$:

\begin{equation}\label{eq:mik_3}
F_{e} = A\left( 1+C \frac{\varphi^2}{D^2}+K\frac{\varphi^4}{D^4}\right) 
\left\lbrace 1- \left\lbrace 1-exp\left[ 1-\cosh \left( \frac{\varphi}{D}\right) \right] \right\rbrace ^{\Gamma} \right\rbrace  
\end{equation}

Testing several dozens of LCs of various types of ESs, we found
that the standard deviation of the fit is typically well below one percent. 
The only minor inconvenience is the existence of a spike (a jump in derivatives) in 
the mid-eclipses for LCs with $\Gamma < 1$.

The contribution of proximity effects $F_{p}(\vartheta)$ should be an
even function symmetric with the phases 0.0 and 0.5, consequently
they can be satisfactorily expressed as a linear combination of $n_{p}$ elementary cosine functions $\cos(2\pi \vartheta)$,
$\cos(4\pi \vartheta)$, $\cos(6\pi \vartheta)$, etc. The even terms are the consequence
of the ellipticity of tidally interacting components, whilst the odd
terms result from the differences between the near and far sides
of components. As a rule we can limit ourselves only to the first
two or three terms in the $F_{p}$ \parencite{russell1952,kallrath1999}. 
The O’Connell effect contribution $F_{c}(\vartheta)$ can be modelled well by a simple sinusoid \parencite{davidge1984, wilsey2009}:

\begin{equation}\label{eq:mik_Fp}
F_{p}  = \sum_{k=n_{e}+1}^{n_{p}+n_{e}} A_{k}\cos \left[ 2\pi(k-n_{e})\vartheta\right]
\end{equation}

\begin{equation}\label{eq:mik_Fc}
F_{c}  = \sum_{k=n_{p}+n_{e}+1}^{n_{c}+n_{p}+n_{e}} A_{k}\cos(2\pi\vartheta)
\end{equation}

where $n_{p}$ is the number of terms in $F_{p}(\vartheta)$: $n_{p} = 0, 1, 2, 3,\ldots $,
$n_{c} = 0$, if the O’Connell asymmetry is not present, otherwise $n_{c} = 1$.

\subsection{Andronov phenomenological model}
The basic function („special shape“) for the eclipse is $H(z)=(1-\left| z\right|^\beta)^{3/2}$, $-1\leq z \geq +1$, where $\beta=C_{5}$ 
is the parameter describing behaviour close to the mid-eclipse (0 – very narrow, 1 – triangular, 2 – parabolic, $\gg2$ – flat).

The complete function includes a TP2 part (a trigonometrical polynomial of the second order), which
approximates three effects: reflection, ellipticity and O’Connell and has 12 parameters, including two for the
corrected initial epoch and the period (Andronov, Tkachenko \& Chinarova, 2016):

\begin{equation}\label{eq:andronov}
\begin{split}
x(\phi)=C_{1}+C_{2}\cos(2\pi(\phi-\phi_{0}))+ C_{3}\sin(2\pi(\phi-\phi_{0}))+C_{4}\cos(4\pi(\phi-\phi_{0}))+ \\
C_{5}\sin(4\pi(\phi-\phi_{0}))+ C_{6}H((\phi-\phi_{0})/C_{8};C_{9})+ C_{7}H((\phi-\phi_{0}-0.5)/C_{8};C_{10})
\end{split}
\end{equation}

In previous works \parencite{andronov2012}, $\phi_{0}=0$ was used but, in \cite{andronov2016}, authors added two parameters
(C11, C12) to correct the initial epoch and the period. Of course, when needed, it is possible to add more
parameters to describe the possible period changes.

Even if more complicated models may be comparable in accuracy, the
smallest possible number of the NAV (“New Algol Variable”)  parameters and their clear physical sense is the advantage of this method. 

